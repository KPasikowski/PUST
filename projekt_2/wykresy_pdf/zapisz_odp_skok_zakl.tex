\documentclass[a4paper,11pt]{article}
\usepackage{pgfplots}
\usepackage{siunitx}
\sisetup{detect-weight,exponent-product=\cdot,output-decimal-marker={,},
	per-mode=symbol,binary-units=true,range-phrase={-},range-units=single}
\SendSettingsToPgf
\usetikzlibrary{pgfplots.groupplots}
\pgfplotsset{compat=1.11}
\usepgfplotslibrary{external}
\tikzexternalize

\textwidth 160mm \textheight 247mm

\pgfplotsset{width=\figurewidth,compat=1.11}
\pgfplotsset{
	tick label style={font=\tiny},
	label style={font=\footnotesize},
	legend style={font=\footnotesize},
	title style={font=\footnotesize}
}


\newcommand{\szer}{16cm}
\newcommand{\wys}{5.6cm}
\newcommand{\odstepionowy}{1.2cm}





\begin{document}

\tikzsetnextfilename{}

\begin{figure}[Odpowiedź obiektu na skokową zmianę zakłócenia]
\tikzsetnextfilename{odp_skok_zakl}
\begin{tikzpicture}
\begin{groupplot}[group style={group size=1 by 2,vertical sep=\odstepionowy},
width=\szer,height=\wys]
%%1
\nextgroupplot
[xmin=1,xmax=200,ymin=-2,ymax=2,
xtick={0, 50, 100, 150, 200},
ytick={-2, -1.5,-1.0, -0.5, 0, 0.5, 1, 1.5, 2},
ylabel={$y(k)$},
xlabel={$k$}]
\addplot[const plot,color=red,semithick] file {zad2/skok_zaklocenia/wyjscie_skok_0.4.txt};
\addplot[const plot,color=blue,semithick] file {zad2/skok_zaklocenia/wyjscie_skok_-0.4.txt};
\addplot[const plot,color=green,semithick] file {zad2/skok_zaklocenia/wyjscie_skok_0.8.txt};
\addplot[const plot,color=purple,semithick] file {zad2/skok_zaklocenia/wyjscie_skok_-0.8.txt};
\nextgroupplot
%%2
[xmin=1,xmax=200,ymin=-1,ymax=1,
xtick={0, 50, 100, 150, 200},
ytick={-0.8,-0.4,0.4,0.8},
ylabel={$z(k)$},
xlabel={$k$}]
\addplot[const plot,color=red,semithick] file {zad2/skok_zaklocenia/zaklocenie_skok_0.4.txt};
\addplot[const plot,color=blue,semithick] file {zad2/skok_zaklocenia/zaklocenie_skok_-0.4.txt};
\addplot[const plot,color=green,semithick] file {zad2/skok_zaklocenia/zaklocenie_skok_0.8.txt};
\addplot[const plot,color=purple,semithick] file {zad2/skok_zaklocenia/zaklocenie_skok_-0.8.txt};

\end{groupplot}
\end{tikzpicture}
\end{figure}

\end{document}
