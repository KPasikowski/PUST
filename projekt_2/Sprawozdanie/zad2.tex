\chapter{Zadanie 2.}
Zar�wno sterowanie, jak i zak?�cenie zosta?y wzbudzone do warto?ci:
\begin{itemize}
\item {\color{green}$ \num{0.8} $}
\item {\color{red}$ \num{0.4} $}
\item {\color{blue}$ \num{-0.4} $}
\item {\color{purple}$ \num{-0.8} $}
\end{itemize}

Odpowiedzi skokowe dla toru wej?cie-wyj?cie wida? na rysunku ~\ref{odp_skok_u} a dla toru zak?�cenie wyj?cie na rysunku ~\ref{odp_skok_z}.

\begin{figure}[b]
\centering
\includegraphics[scale=1]{../wykresy_pdf/odp_skok_ster.pdf}
\caption {Odpowied? skokowa toru wej?cie-wyj?cie}
\label{odp_skok_u}
\end{figure}

\begin{figure}[b]
\centering
\includegraphics[scale=1]{../wykresy_pdf/odp_skok_zakl.pdf}
\caption {Odpowied? skokowa toru zak?�cenie-wyj?cie}
\label{odp_skok_z}
\end{figure}

Charakterystyk? statyczn? $ y(u,z) $ procesu przedstawia wykres ~\ref{char_stat}

\begin{figure}[b]
\centering
\includegraphics[scale=1]{../wykresy_pdf/char_stat.pdf}
\caption {Charakterystyka statyczna procesu}
\label{char_stat}
\end{figure}

Z charakterystyki wida?, ?e w?a?ciwo?ci statyczne procesu s? liniowe. W?a?ciwo?ci dynamiczne r�wnie?, st?d mo?emy obliczy? wzmocnienie statyczne toru wej?cie-wyj?cie:
$$ K = \frac{\Delta y}{\Delta u} = \num{1.8737} $$
oraz wzmocnienie statyczne toru zak?�cenie-wyj?cie:
$$ K = \frac{\Delta y}{\Delta z} = \num{1.0837} $$