\documentclass[a4paper,titlepage,11pt,twosides,floatssmall]{mwrep}
\usepackage[left=2.5cm,right=2.5cm,top=2.5cm,bottom=2.5cm]{geometry}
\usepackage[OT1]{fontenc}
\usepackage{polski}
\usepackage{amsmath}
\usepackage{amsfonts}
\usepackage{amssymb}
\usepackage{graphicx}
\usepackage{url}
\usepackage{tikz}
\usetikzlibrary{arrows,calc,decorations.markings,math,arrows.meta}
\usepackage{rotating}
\usepackage[percent]{overpic}
\usepackage[cp1250]{inputenc}
\usepackage{xcolor}
\usepackage{pgfplots}
\usetikzlibrary{pgfplots.groupplots}
\usepackage{listings}
\usepackage{matlab-prettifier}
\usepackage{siunitx}
\usepackage{placeins}
\definecolor{szary}{rgb}{0.95,0.95,0.95}
\sisetup{detect-weight,exponent-product=\cdot,output-decimal-marker={,},per-mode=symbol,binary-units=true,range-phrase={-},range-units=single}

%konfiguracje pakietu listings
\lstset{
	backgroundcolor=\color{szary},
	frame=single,
	breaklines=true,
}
\lstdefinestyle{customlatex}{
	basicstyle=\footnotesize\ttfamily,
	%basicstyle=\small\ttfamily,
}
\lstdefinestyle{customc}{
	breaklines=true,
	frame=tb,
	language=C,
	xleftmargin=0pt,
	showstringspaces=false,
	basicstyle=\small\ttfamily,
	keywordstyle=\bfseries\color{green!40!black},
	commentstyle=\itshape\color{purple!40!black},
	identifierstyle=\color{blue},
	stringstyle=\color{orange},
}
\lstdefinestyle{custommatlab}{
	captionpos=t,
	breaklines=true,
	frame=tb,
	xleftmargin=0pt,
	language=matlab,
	showstringspaces=false,
	%basicstyle=\footnotesize\ttfamily,
	basicstyle=\scriptsize\ttfamily,
	keywordstyle=\bfseries\color{green!40!black},
	commentstyle=\itshape\color{purple!40!black},
	identifierstyle=\color{blue},
	stringstyle=\color{orange},
}

%wymiar tekstu (bez �ywej paginy)
\textwidth 160mm \textheight 247mm

%ustawienia pakietu pgfplots
\pgfplotsset{
tick label style={font=\scriptsize},
label style={font=\small},
legend style={font=\small},
title style={font=\small}
}

\def\figurename{Rys.}
\def\tablename{Tab.}

%konfiguracja liczby p�ywaj�cych element�w
\setcounter{topnumber}{0}%2
\setcounter{bottomnumber}{3}%1
\setcounter{totalnumber}{5}%3
\renewcommand{\textfraction}{0.01}%0.2
\renewcommand{\topfraction}{0.95}%0.7
\renewcommand{\bottomfraction}{0.95}%0.3
\renewcommand{\floatpagefraction}{0.35}%0.5

\begin{document}
\frenchspacing
\pagestyle{uheadings}

%strona tytu�owa
\title{\bf Sprawozdanie z projektu nr 4, zadanie nr 6\vskip 0.1cm}
\author{Mateusz Koro�, Ksawery Pasikowski, Mateusz Morusiewicz}
\date{2017}

\makeatletter
\renewcommand{\maketitle}{\begin{titlepage}
\begin{center}{\LARGE {\bf
Wydzia� Elektroniki i Technik Informacyjnych}}\\
\vspace{0.4cm}
{\LARGE {\bf Politechnika Warszawska}}\\
\vspace{0.3cm}
\end{center}
\vspace{5cm}
\begin{center}
{\bf \LARGE Projektowanie uk�ad�w sterowania\\ (projekt grupowy) \vskip 0.1cm}
\end{center}
\vspace{1cm}
\begin{center}
{\bf \LARGE \@title}
\end{center}
\vspace{2cm}
\begin{center}
{\bf \Large \@author \par}
\end{center}
\vspace*{\stretch{6}}
\begin{center}
\bf{\large{Warszawa, \@date\vskip 0.1cm}}
\end{center}
\end{titlepage}
}
\makeatother

\maketitle
\tableofcontents

\chapter{Zad. 1}
Poprawno�� punktu pracy zosta�a udowodniona poprzez sprawdzenie, czy obiekt, b�d�cy w punkcie pracy, pozostanie w nim, je�li warto�ci sterowania pozostanie taka sama. Zosta�o to wykonane za pomoc� komendy:
\begin{lstlisting}[style=Matlab-editor]
y_ust = symulacja_obiektu6y(0, 0, 0, 0)
\end{lstlisting}
Co da�o wynik [0,0], co dowodzi, �e punktem pracy rzeczywi�cie jest punkt $ u=y=0 $.

\chapter{Zad. 2}
Sterowanie zosta�o wzbudzone do warto�ci:
\begin{itemize}
	\item {\color{green}$ \num{0.4} $}
	\item {\color{purple}$ \num{0.8} $}
	\item {\color{red}$ \num{-0.4} $}
	\item {\color{blue}$ \num{-0.8} $}
\end{itemize}

Z charakterystyki wida�, �e w�a�ciwo�ci statyczne procesu nie s� liniowe, dynamiczne r�wnie�.

\chapter{Zad. 3}
\section{PID}
\begin{lstlisting}[style=Matlab-editor]
function [ y, u, E, yzad ] = policzPID( Kp_, Ti_, Td_, Kk_)
U_min = -1;
U_max = 1;

Kp = Kp_;
Ti = Ti_;
Td = Td_;

Kk = Kk_;
Tp = 0.5;

r2 = (Kp * Td) / Tp ;
r1 = Kp * ( (Tp/(2*Ti)) - 2*(Td/Tp) - 1 ) ;
r0 = Kp * ( 1 + Tp/(2*Ti) + Td/Tp ) ;

%warunki poczatkowe
u(1:11) = 0 ;
y(1:11) = 0 ;
e(1:11) = 0 ;
index = 1;
yzads = [-1 -2.5 -1 0.06];
yzad = yzads(index);
yzadVec(1:Kk) = yzad;
% u(1:11) = 0.34 ;
% y(1:11) = 0.073 ;
% e(1:11) = 0 ;
% index = 1;
% yzads = [0.084];
% yzad = yzads(index);
% yzadVec(1:Kk) = yzad;


% glowna petla symulacji
for k = 7 : Kk
if mod(k,200) == 0
index = index + 1;
if index > length(yzads)
index = length(yzads);
end
yzad = yzads(index);
end
yzadVec(k) = yzad;

y(k) = symulacja_obiektu6y(u(k-5), u(k-6), y(k-1), y(k-2));

e(k) = yzad - y(k) ;

u(k) = r2 * e(k-2) + r1 * e(k-1) + r0 * e(k) + u(k-1) ;

if u(k) > U_max
u(k) = U_max;
elseif u(k) < U_min
u(k) = U_min;
end


end

E = (yzadVec - y) * (yzadVec - y)';

yzad = zeros(1, Kk);
yzad(1, :) = yzadVec;

end
\end{lstlisting}

\section{DMC}
\begin{lstlisting}[style=Matlab-editor]
function [ Y, U, E, yzadVec ] = policzDMC( D_, N_, Nu_, lambda_, Kk_)

D=D_;
N=N_;
Nu=Nu_;
lambda=lambda_;

% testowanie i dobieranie parametr?w z zadania 6 i 7
s = ...
 load('wykresy_pliki/zad6/odpowiedzi/wyjscie_skok_-1_-0.75.txt');
s = s(:, 2);


Upp=0;
Ypp=0;
Umin=-1;
Umax=1;

% testowanie i dobieranie parametr?w z zadania 6 i 7
index = 1;
yzads = [-2.264];
yzad = yzads(index);
yzadVec(1:Kk_) = yzad;
Yzad = yzadVec - Ypp;

% index = 1;
% yzads = [-1 -2.5 -1 0];
% yzad = yzads(index);
% yzadVec(1:Kk_) = yzad;
% Yzad = yzadVec - Ypp;
%inicjalizacja sta?ych
kk=Kk_;
%DMC
%----------------------------------------------------------------
M=zeros(N,Nu);
for i=1:N
for j=1:Nu
if (i>=j)
M(i,j)=s(i-j+1);
end;
end;
end;

MP=zeros(N,D-1);
for i=1:N
for j=1:D-1
if i+j<=D
MP(i,j)=s(i+j)-s(j);
else
MP(i,j)=s(D)-s(j);
end;      
end;
end;

% Obliczanie parametr?w regulatora

I=eye(Nu);
K=((M'*M+lambda*I)^-1)*M';
ku=K(1,:)*MP;
ke=sum(K(1,:));

% U(1:kk)=Upp;
% Y(1:kk)=Ypp;

% testowanie i dobieranie parametr?w z zadania 6 i 7
U(1:kk)=-0.96;
Y(1:kk)=-3.187;

e=zeros(1,kk);

u=U-Upp;
y=Y-Ypp;
umax = Umax - Upp;
umin = Umin - Upp;

deltaup=zeros(1,D-1);

for k=7:kk
if mod(k,200) == 0
index = index + 1;
if index > length(yzads)
index = length(yzads);
end
yzad = yzads(index);
end
yzadVec(k) = yzad;
Yzad(k) = yzadVec(k) - Ypp;

%symulacja obiektu
Y(k)= symulacja_obiektu6y(U(k-5), U(k-6), Y(k-1), Y(k-2));
y(k) = Y(k) - Ypp;
%uchyb regulacji
e(k)=Yzad(k) - y(k);

% Prawo regulacji
deltauk=ke*e(k)-ku*deltaup';

for n=D-1:-1:2
deltaup(n)=deltaup(n-1);
end
deltaup(1)=deltauk;
u(k)=u(k-1)+deltaup(1);

if u(k)>umax
u(k)=umax;
elseif u(k)<umin
u(k)=umin;
end
U(k)=u(k)+Upp;

end

%obliczenie b??du
E=0;
for k=1:kk
E=E+((Yzad(k)-Y(k))^2);
end 

end
\end{lstlisting}

\chapter{Zad. 4}
Regulatory oceniane by�y na podstawie wykres�w oraz warto�ci wska�nika jako�ci:
$$ E= \sum_{k=1}^{k_{konc}}\sum_{m=1}^{2}(y_{m}^{zad}(k)-y(k))^{2}$$
Najlepszymi znalezionymi warto�ciami parametr�w regulatora PID by�y: $K_{p}=\num{0.5}\quad T_{i}=10\quad T_{d}=\num{0.5} $, natomiast najlepsz� warto�ci� parametru $ \lambda $ dla regulatora DMC by�o $ 1000 $.
\end{document}


