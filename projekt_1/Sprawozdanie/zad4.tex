\chapter{Zadanie 4.}
\section{PID}
Program, realizuj�cy symulacj� algorytmu PID wygl�da nast�puj�co:
\begin{lstlisting}[style=Matlab-editor]
Kp = 3 ;
Ti = 10 ;
Td = 3.2 ;

r2 = (Kp * Td) / Tp ;
r1 = Kp * ( (Tp/(2*Ti)) - 2*(Td/Tp) - 1 ) ;
r0 = Kp * ( 1 + Tp/(2*Ti) + Td/Tp ) ;

% warunki poczatkowe
u(1:11) = Upp ;
U(1:11) = Upp ;
y(1:11) = Ypp ;
y2(1:11) = Ypp ;
e(1:11) = 0 ;
delta_u = 0;
index = 1;
yzads = [1.05 0.8 1.1 0.9];
yzad = yzads(index);   %skok wartosci zadanej
yzad2 = yzad - Ypp;
yzadVec(1:Kk) = yzad;

% glowna petla symulacji
for k = 12 : Kk
    if mod(k,400) == 0
        index = index + 1;
        if index > length(yzads)
            index = length(yzads);
        end
        yzad = yzads(index);
        yzad2 = yzad - Ypp;
    end
    yzadVec(k) = yzad;
    
    y(k) = symulacja_obiektu4Y(U(k-10), U(k-11), y(k-1), y(k-2)) ;
    
    y2(k) = y(k) - Ypp;
    e(k) = yzad2 - y2(k) ;
    
    u(k) = r2 * e(k-2) + r1 * e(k-1) + r0 * e(k) + u(k-1) ;
    
    delta_u = u(k) - u(k-1);
    
    if delta_u > dU_max
        delta_u = dU_max;
    elseif delta_u < -dU_max
        delta_u = -dU_max;
    end
    
    u(k) = u(k-1) + delta_u;
    
    if u(k) > U_max - Upp
        u(k) = U_max - Upp;
    elseif u(k) < U_min - Upp
        u(k) = U_min - Upp;
    end
    
     U(k) = u(k) + Upp;
end
\end{lstlisting} 
\section{DMC}
Natomiast realizacja DMC w wersji analitycznej:
\begin{lstlisting}[style=Matlab-editor]
s = policz_odp_skok();

D = policzHorDynamiki(s) ;   %horyzont dynamiki
% N = D ;   %horyzont predykcji
% Nu = D ;  %horyzont sterowania 
% lambda = 20;
yzads = [1.05 0.8 1.1 0.9];
index = 1;
yzad = yzads(index);   %skok wartosci zadanej
yzadVec(1:Kk) = yzad;

%sygnal sterujacy
u = Upp + zeros(1,N) ;
U = Upp + zeros(1,N);
%uchyb
e = zeros(1,N) ;
%wyjscie ukladu
y = zeros(1,Kk) + Ypp ;

%obliczanie odpowiedzi skokowej
du = (zeros(1,D-1))' ;

M = zeros(N, Nu) ;
for i = 1:N
    for j = 1:Nu
        if (i-j+1 > 0)
            M(i,j) = s(i-j+1) ;
        else
            M(i,j) = 0 ;
        end
    end
end

Mp = zeros(N, D-1) ;
for i = 1:N
    for j = 1:(D-1)
        if(i+j <= N)
            Mp(i,j) = s(i+j) - s(j) ;
        else
            Mp(i,j) = s(N) - s(j) ;
        end
    end
end

K = (M'*M + lambda*eye(Nu))^-1 * M' ;

%liczenie ke
ke = 0;
for i = 1:N
    ke = ke + K(1, i);
end

kju = K(1,:)*Mp;
y2 = zeros(Kk, 1);

for k = 12:Kk

    if mod(k,400) == 0
        index = index + 1;
        if index > length(yzads)
            index = length(yzads);
        end
        yzad = yzads(index);
    end
    yzadVec(k) = yzad;
   
    y(k) = symulacja_obiektu4Y(U(k-10), U(k-11), y(k-1), y(k-2)) ;
    
    sum = 0;    %suma potrzebna do obliczenia skladowej swobodnej
    for j = 1:D-1
        if(k-j > 0)
            sum = sum + kju(j)*du(k-j);
            %w innym przypadku du = 0 wiec sum sie nie zmienia
        end
    end
    y2(k) = y(k) - Ypp;
    yzad2 = yzad - Ypp;
    du(k) = ke * (yzad2-y2(k)) - sum ;
    
    % --- sprawdzenie, czy przyrost znajduje sie w ograniczeniach  ---
    if du(k) > dU_max
        du(k) = dU_max;
    elseif du(k) < -dU_max
        du(k) = -dU_max;
    end
    
    u(k) = u(k-1) + du(k);
    
    if u(k) > U_max - Upp
        u(k) = U_max - Upp;
    elseif u(k) < U_min - Upp
        u(k) = U_min - Upp;
    end
    
    U(k) = u(k) + Upp;
    
end
\end{lstlisting}

Ograniczenia zar�wno warto�ci jak i szybko�ci wzrostu sygna�u steruj�cego w obu regulatorach zosta�y uwzgl�dnione poprzez rzutowanie po obliczeniu sterowania przez algorytm, natomiast przesuni�cie do punktu pracy zosta�o wykonane za pomoc� odj�cia warto�ci $ Y_{pp} $ przed algorytmem regulacji i dodania $ U{pp} $ do obliczonego sterowania.